% Software Libraries

\cvitem{2015--Present}{\textbf{Spack}\tabto{6em}\faStar{} 3.7K\tabto{11em}\faCodeFork{} 2.1K\tabto{16em}\url{https://github.com/spack/spack}}

\cvlistitem{Supercomputing PACKage manager}
\cvlistitem{Core developer, maintainer, top contributor}

\cvitem{2021--Present}{\textbf{TorchGeo}\tabto{6em}\faStar{} 2.0K\tabto{11em}\faCodeFork{} 240\tabto{16em}\url{https://github.com/microsoft/torchgeo}}

\cvlistitem{PyTorch domain library for geospatial data}
\cvlistitem{Creator, core developer, maintainer, top contributor}

\cvitem{2022--Present}{\textbf{Rtree}\tabto{6em}\faStar{} 570\tabto{11em}\faCodeFork{} 130\tabto{16em}\url{https://github.com/Toblerity/rtree}}

\cvlistitem{R-tree spatial index for Python GIS}
\cvlistitem{Core developer, maintainer, top 5 contributor}